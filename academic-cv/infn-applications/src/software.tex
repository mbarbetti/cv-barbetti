\newcommand{\software}[1]
  {\normalsize \color{hlcolor-0} \textbf{#1}}

\newcommand{\pkginfo}[1]
  {\small \color{hlcolor-1} {\ul{\textsc{Languages}:} #1}}
  
\newcommand{\pkgdesc}[1]
  {\normalsize \color{maincolor} \emph{#1}}


\section*{Contributions to Open Source Software}
\begin{cvcontent}
  %%%
  %%%  fast-simulation
  %%% =================
  \software{LHCb Fast Simulation}
  \\ [0.5mm]
  \pkginfo{Python, C}
  \\ [1.5mm]
  \pkgdesc{Set of tools and Python modules to prepare/process data, 
  train Machine Learning models, and export them as C files to enable 
  the integration within the LHCb software stack. Training scripts
  are designed to build parameterizations for the Particle 
  Identification system of the LHCb experiment.}
  \\ [1.5mm]
  \begin{itemize*}[label=\textcolor{iconcolor}{\textbullet}]
    \item \texttt{mbarbetti/lb-pidsim-train} \hfill
      \href{https://pypi.org/project/lb-pidsim-train}{\faPython}
      \href{https://github.com/mbarbetti/lb-pidsim-train}{\faGithub}
    \\ [0.5mm]
    \item \texttt{landerlini/scikinC} \hfill
      \href{https://pypi.org/project/scikinC}{\faPython}
      \href{https://github.com/landerlini/scikinC}{\faGithub}
  \end{itemize*}
  \\ [4mm]
  %%%
  %%%  optimization
  %%% ==============
  \software{Hyperparameter Optimization in the Cloud}
  \\ [0.5mm]
  \pkginfo{Python, JavaScript, HTML}
  \\ [1.5mm]
  \pkgdesc{Set of Python modules to deliver and access a cloud-based
  service for hyperparameter optimization through HTTP requests. 
  Computing nodes contribute to optimization campaigns via token-based
  authentication, and a web dashboard allows to monitor the status of
  past and ongoing optimization studies.}
  \\ [1.5mm]
  \begin{itemize*}[label=\textcolor{iconcolor}{\textbullet}]
    \item \texttt{landerlini/hopaas} \hfill
      \href{https://hopaas.cloud.infn.it}{\faGlobe}
    \\ [0.5mm]
    \item \texttt{landerlini/hopaas\_client} \hfill
      \href{https://github.com/landerlini/hopaas_client}{\faGithub}
    \\ [0.5mm]
    \item \texttt{mbarbetti/optunapi} \hfill
      \href{https://pypi.org/project/optunapi}{\faPython}
      \href{https://github.com/mbarbetti/optunapi}{\faGithub}
  \end{itemize*}
  \\ [4mm]
  %%%
  %%%  radiomics
  %%% ===========
  \software{Radiomics and Medical Physics}
  \\ [0.5mm]
  \pkginfo{Python, Jupyter Notebook}
  \\ [1.5mm]
  \pkgdesc{Set of notebooks and Python scripts for processing data,
  and for training Machine Learning models powered by radiomic features 
  to infer the histological type of mediastinal bulky lymphoma.}
  \\ [1.5mm]
  \begin{itemize*}[label=\textcolor{iconcolor}{\textbullet}]
    \item \texttt{mbarbetti/lymphoma-classification} \hfill
      \href{https://github.com/mbarbetti/lymphoma-classification}{\faGithub}
  \end{itemize*}
  \\ [4mm]
  %%%
  %%%  others
  %%% ========
  \software{Others}
  \\ [0.5mm]
  \pkginfo{Python, JavaScript, HTML, Jupyter Notebook}
  \\ [1.5mm]
  \pkgdesc{List of other personal contributions to open-source software.
  It includes a Jekyll theme for academic websites, a Python package to
  implement generative models in TensorFlow, and a repository containing
  notebooks and code for INFN education events.}
  \\ [1.5mm]
  \begin{itemize*}[label=\textcolor{iconcolor}{\textbullet}]
    \item \texttt{alshedivat/al-folio} \hfill
      \href{https://hub.docker.com/r/amirpourmand/al-folio}{\faDocker}
      \href{https://alshedivat.github.io/al-folio}{\faGlobe}
      \href{https://github.com/alshedivat/al-folio}{\faGithub}
    \\ [0.5mm]
    \item \texttt{mbarbetti/tf-gen-models} \hfill
      \href{https://pypi.org/project/tf-gen-models}{\faPython}
      \href{https://github.com/mbarbetti/tf-gen-models}{\faGithub}
    \\ [0.5mm]
    \item \texttt{landerlini/mlinfn-advanced-hackathon} \hfill
      \href{https://github.com/landerlini/mlinfn-advanced-hackathon}{\faGithub}
  \end{itemize*}
\end{cvcontent}