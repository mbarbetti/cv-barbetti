\section*{Research Activities}
\begin{cvcontent}
  I am a third-year Ph.D. student in Smart Computing at the 
  University of Florence. My research activity is mainly 
  dedicated to the development of \textsc{Lamarr} 
  \ref{item:lamarr-ichep}, the new LHCb \emph{ultra-fast 
  simulation} framework. \textsc{Lamarr} consists of a 
  pipeline of modular parameterizations designed to replace
  both the physics simulation and the reconstruction steps.
  Most of the parameterizations rely on Machine Learning 
  algorithms \ref{item:gan-pm}, such as \emph{generative
  models} that have proven to be able to well reproduce 
  the detailed simulation \ref{item:gan-acat}.
  \\
  As an active developer for \textsc{Lamarr}, I am responsible 
  for the development of parameterizations for the LHCb 
  PID system. Obtaining high-quality distributions for 
  such class of models is a non-trivial task and requires 
  intensive \emph{hyperparameter optimization} studies.
  Thus, Bayesian strategies have been explored to run
  optimization campaigns \emph{in parallel}, and exploit
  opportunistic GPU resources. In this context, I have
  developed Hopaas \ref{item:hopaas}, a \emph{cloud service}
  that allows to orchestrate optimization studies across 
  multiple computing instances via simple HTTP requests.
  Hopaas allows to obtain performing parameterizations
  that can easily replace specific modules without 
  recompiling the whole \textsc{Lamarr} pipeline using 
  \texttt{scikinC} \ref{item:scikinc}, a tool able to
  transpile Machine Learning models to C files.
  \\
  Besides my involvement within the \textsc{Lamarr}
  development, I have also contributed to apply Machine 
  Learning techniques to Medical Physics. In particular,
  I have developed a Machine Learning model powered by
  Radiomics to infer the \emph{histological type} of 
  mediastinal bulky lymphoma masses \ref{item:phys-med}.
  \\
  My research activity fits perfectly with what carried
  out by the LHCb Florence Group that plays a key role
  with \emph{detector simulation} and \emph{cosmic ray 
  physics}. State-of-the-art techniques are applied 
  successfully both for fast simulation and physics 
  analysis where, in particular, the LHCb Florence Group
  is author of a novel approach based on Machine Learning 
  to model the PID response to antiprotons by LHCb in
  fixed-target mode \ref{item:smog-pub}.
\end{cvcontent}