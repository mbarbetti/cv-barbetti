\section*{Research Activities}
\begin{cvcontent}
  \onehalfspacing
  Since 2019, I am member of the LHCb experiment at CERN's Large 
  Hadron Collider (LHC). In this context, I had the opportunity to 
  focus on some of the most challenging computing developments for 
  the next-generation High Energy Physics experiments, including the 
  application of advanced \emph{Deep Learning} techniques to large 
  samples of proton-proton collision data and their training and 
  deployment in a world-wide distributed environment.
  \\
  In the same period, in collaboration with the Careggi University 
  Hospital in Florence, 
  I also had the opportunity to contribute to the research in Medical 
  Physics by applying \emph{Machine Learning} techniques to unpublished 
  radiomic\footnote{In the field of medicine, \emph{Radiomics} is a new 
  field of image analysis in which digital medical images are converted 
  into quantitative high-dimensional data.} datasets.
  \\
  I am currently preparing my Ph.D. thesis in Smart Computing at 
  the University of Florence with original contributions on 
  LHCb Simulation and Radiomics.
  \\ [2mm]
  My research activity is mainly devoted to the development of a novel 
  LHCb \emph{ultra-fast simulation} framework, called \textsc{Lamarr}
  \ref{pub:lamarr-ichep} \ref{pub:lamarr-acat}, that I have presented
  during the international workshop \hyperlink{acat-2022}{[ACAT22]} 
  and more recently at the international conference \hyperlink{chep-2023}
  {[CHEP23]}. \textsc{Lamarr} consists of a pipeline of modular 
  parameterizations designed to replace 
  both the physics simulation and the reconstruction steps. Most of the 
  parameterizations rely on Machine Learning algorithms \ref{pub:gan-pm},
  such as \emph{generative models}, that were demonstrated to be 
  able to well reproduce the distributions obtained from the 
  detailed simulation~\ref{pub:gan-acat}. Part of my Master Thesis
  was devoted to develop a prototype of such parameterizations: this
  preliminary study was honored with the national INFN award
  ``Giulia Vita Finzi'' for the best thesis on computing and networks.
  \\ [2mm]
  I am an active developer within the LHCb Simulation team, with
  the responsibility for the parameterization of the LHCb Particle 
  Identification (PID) system. Generative Deep Neural Networks
  succeed in reproducing the analysis-level classifiers defined 
  by the PID reconstruction algorithms, showing promising results,
  which I presented at national \hyperlink{sif-2021}{[SIF21]} 
  and international \hyperlink{ltd}{[LTD]} conferences. 
  Parameterizing with high accuracy the PID multivariate 
  distributions requires training such specialized models
  in GPU-intensive \emph{hyperparameter optimization} campaigns.
  \\
  To this end, to profit from multiple computing resources made 
  opportunistically available to the LHCb Collaboration, I proposed 
  a client-server approach with a centralized service defining the 
  optimization procedure in multiple instances running in parallel 
  on different providers. The approach was validated on resources made 
  available to LHCb by CERN via the \hyperlink{cloud-bank}{[CloudBank]} 
  initiative or by INFN and CINECA granting access to the Marconi 100 
  supercomputer. During the last few months, I have been converting the 
  validated prototype into an INFN Cloud service, named Hopaas 
  \ref{pub:hopaas}, featuring a web dashboard and a token-based user 
  authentication procedure integrated to the INFN user identity management 
  tools. I have presented Hopaas during the international 
  workshop \hyperlink{acat-2022}{[ACAT22]}.
  \\ [2mm]
  To enable the deployment of the optimized models in the LHCb software 
  stack, running in multiple data centers around the world connected 
  via the LHCb Computing Grid, I contributed to the development of 
  \texttt{scikinC} \ref{pub:scikinc}, a tool able to translate 
  Machine Learning models into C files that, once compiled, 
  are distributed with the CernVM File System and dynamically 
  linked to the LHCb applications.
  \\ [2mm]
  My research is strongly connected to the data analysis effort within 
  the LHCb group in Florence, exploiting advanced statistical treatment 
  of the PID classifiers with Machine Learning techniques to shed light 
  on the nature and the antiprotons observed in primary cosmic rays 
  \ref{pub:smog-pub}.
  \\ [2mm]
  As part of the research activities for my Ph.D., I am exploring the 
  application of Machine Learning techniques to Medical Physics to 
  develop solutions suitable for the emerging ``personalized medicine''. 
  In particular, I am working to build a Machine Learning model able 
  to infer the histological type of mediastinal bulky lymphomas, one 
  of the most common hematologic cancers, only using radiomic features 
  drawn from PET images \ref{pub:phys-med}. The aim of the study is to 
  understand how PET radiomic features may predict lymphoma histology 
  and in the future support its diagnosis, offering non-invasive solutions 
  tailored to the single patient. I presented the promising preliminary 
  results at national~\hyperlink{sif-2022}{[SIF22]} and 
  international \hyperlink{ecmp-2022}{[ECMP22]} conferences.
  \\ [2mm]
  In 2020 I joined the \emph{ML-INFN} project, a national initiative 
  of INFN to coordinate the effort toward the widespread application 
  of Machine Learning technologies to research activities. In 2022 I 
  helped in organizing the Third ML-INFN Hackathon for which I gave
  two lectures, one about deep generative models and the other about
  Bayesian hyperparameter optimization \hyperlink{mlinfn-2022}{[ML\_INFN22]}.
\end{cvcontent}