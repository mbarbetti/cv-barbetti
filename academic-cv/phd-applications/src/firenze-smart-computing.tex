\documentclass[a4, 11pt]{report}
\usepackage[utf8x]{inputenc}
\usepackage{graphicx}

\usepackage[yyyymmdd]{datetime}
\renewcommand{\dateseparator}{-}

\usepackage{hyperref}
\newcommand{\mail}[1]{\href{mailto:#1}{\texttt{#1}}}

\usepackage{soul}
%\usepackage{ulem}
\usepackage{titlesec}
\titlespacing\section{0pt}{18pt}{6pt}

\usepackage{geometry}
\geometry{a4paper, top=2cm, bottom=2cm,
		left=2cm, right=2cm, heightrounded,
		bindingoffset=0cm}
		
\usepackage{lastpage}
\usepackage{fancyhdr}
\pagestyle{fancy}
\renewcommand{\headrulewidth}{0pt}
\cfoot{}
\rfoot{\textit{\small page \thepage\ of \pageref*{LastPage}}}


\author{Matteo Barbetti}
\title{Curriculum Vitae}


\begin{document}

    \begin{center}
        \begin{Large}
            \textbf{Matteo Barbetti}
        \end{Large}\\[1mm]
        \textsc{PhD Student in Smart Computing}
    \end{center}
    
    \vskip 3mm
    
    \section*{Personal information}
        \begin{minipage}{0.45\textwidth}
            \begin{tabular}{rl}
                \textit{Full name:}   & Matteo Barbetti\\
                \textit{Nationality:} & {Italian}\\
                \textit{Birth-place:} & {Terni (Italy)}\\
                \textit{Birth-date:}  & {1994-05-06}\\          
            \end{tabular}
        \end{minipage}
        \hfill
        \begin{minipage}{0.5\textwidth}
            \begin{tabular}{rl}
                \textit{e-mail:}      & \mail{matteo.barbetti.94@gmail.com}\\
                \textit{Telephone:}   & +39\,\,347\,8595002\\
                \textit{Address:}     & Via Vittorio Emanuele II, 193\\
                                      & Firenze (FI) -- 50134, Italy\\
            \end{tabular}
        \end{minipage}

    \vskip 9mm

    \noindent
    I have completed the Master degree in \emph{Particle Physics} at the University of Florence in June 2020. I have developed a \emph{curriculum} of studies oriented to Physics Data Analysis with a strong focus on modern Computing Technologies, including in my study program the classes of \emph{Algorithms and Data Structures} and \emph{Machine Learning} from the Master course in Computing Engineering. 

    \vskip 3mm
    
    \noindent
    In September 2017 I discussed a Bachelor Thesis focused on Data Analysis work in High Energy Physics: the study of the decays of \textit{charmonium} states into purely hadronic final states as reconstructed with the LHCb experiment at CERN. A scholarship funded by the University of Florence offered to me the opportunity of carrying out part of the activities at CERN.

    \vskip 3mm
    
    \noindent
    In June 2020 I discussed a Master Thesis focused on the development of deep generative models to simulate the high-level response of the Particle Identification system of the LHCb detector. The trained models were then integrated within a new simulation framework designed to outperform the traditional simulation programs, taking full advantage of modern softwares for parallel computing and exploiting efficiently hardware accelerators.
    A fellowship funded by INFN has offered me the opportunity of passing three months at CERN, and forging new relationships within a stimulant and international environment.
    
    \vskip 3mm
    
    \noindent
    I am a member of AISF, the Italian Association of Physics Students, which promotes and stimulates expanding the professional networks of its members in a national and international context. I held the position of President of the Florence Local Committee until May 2019; now I am the National Secretary of AISF.
    
    \vskip 9mm
    
    % EDUCATION
    \section*{Education}
        \begin{tabular*}{\textwidth}{rl @{\extracolsep{\fill}} r}
            \textit{Laurea Magistrale}  
            & {\textbf {Università degli Studi di Firenze}}
            & {\small 2017 -- 2020}\\
            \textit{in Fisica (MSc)}
            & {\small Graduation on June 2020 with score 110/110 \textit{cum laude}}\\ [1.5mm]
            %& {\small \textit{Curriculum: Particle Physics}}\\ [1.5mm]
            & \multicolumn{2}{l}{``\textit{Techniques for parametric simulation with deep neural networks and}}\\
            & \multicolumn{2}{l}{\textit{implementation for the LHCb experiment at CERN and its future upgrades}''}\\
            & {\small Thesis Advisor: Lucio Anderlini}\\ [3mm]
            
            \textit{Laurea Triennale}   
            & {\textbf {Università degli Studi di Firenze}}
            & {\small 2013 -- 2017}\\
            \textit{in Fisica (BSc)}    
            & {\small Graduation on September 2017 with score 110/110}\\ [1.5mm]
            & \multicolumn{2}{l}{``\textit{Study of the charmonium resonances in $B^+ \to p \bar{p} K^+$ and $B^+ \to p \bar{p} \gamma K^+$}}\\
            & \multicolumn{2}{l}{\textit{ decays with the LHCb experiment at CERN}''}\\
            & {\small Thesis Advisor: Lucio Anderlini}\\ [3mm]
                                        
            \textit{Previous education} 
            & {\textbf {Liceo Scientifico G. Galilei di Terni}}
            & {\small 2008 -- 2013}\\
            & {\small Graduation on July 2013 with score 100/100 \textit{cum laude}}
        \end{tabular*}
        
    \pagebreak

    % MASTER DEGREE COURSES
    \section*{Master degree classes}
        \begin{tabular*}{\textwidth}{l @{\extracolsep{\fill}} r}
            {\ul {Algoritmi e strutture dati}} & {\ul {Analisi dati in fisica subnucleare}}\\
            {\small \emph {Algorithm theory and analysis (sorting problem).}} & {\small \emph {HEP statistics methods and MC simulations.}}\\
            {\small \emph {Data structures and graph algorithms.}} & {\small \emph {Data analysis with ROOT and RooFit.}}\\
            {\small S. Marinai} & {\small P. Lenzi, N. Mori}\\
            {\small [6 \textsc{cfu}] 30/30, 2018-09-14} & {\small [6 \textsc{cfu}] 27/30, 2018-07-19}\\ [3mm]

            {\ul {Atomi, molecole e fotoni}} & {\ul {Complementi di fisica subnucleare}}\\
            {\small \emph {Atomic and molecular spectroscopy. Radiation-}} & {\small \emph {Discrete symmetries. S-matrix theory.}}\\
            {\small \emph {matter interaction. Cooling and trapping atoms.}} & {\small \emph {Decay width and scattering cross section.}}\\
            {\small G. M. Tino} & {\small E. Iacopini}\\
            {\small [6 \textsc{cfu}] 29/30, 2020-02-21} & {\small [6 \textsc{cfu}] 30/30 \textit{cum laude}, 2019-06-17}\\ [3mm]

            {\ul {Cosmologia}} & {\ul {Fisica delle alte energie}}\\
            {\small \emph {Cosmology theory and thermal history of universe.}} & {\small \emph {Gauge theory. Electroweak interaction.}}\\
            {\small \emph {CMB perturbations and fluctuations.}} & {\small \emph {SM and CP violation. Collider physics.}}\\
            {\small A. Marconi} & {\small G. Barbagli, C. Civinini}\\
            {\small [6 \textsc{cfu}] 30/30 \textit{cum laude}, 2018-03-14} & {\small [6 \textsc{cfu}] 28/30, 2019-07-15}\\ [3mm]

            {\ul {Fisica nucleare}} & {\ul {Fisica subnucleare}}\\
            {\small \emph {Nuclear force and two-nucleon systems.}} & {\small \emph {Group theory. Relativistic kinematics.}}\\
            {\small \emph {Nuclear structure and reactions ($\alpha$, $\beta$, $\gamma$).}} & {\small \emph {Elementary particle and weak interaction.}}\\
            {\small M. Bini} & {\small E. Iacopini}\\
            {\small [6 \textsc{cfu}] 30/30, 2018-02-12} & {\small [6 \textsc{cfu}] 29/30, 2019-04-16}\\ [3mm]

            {\ul {Fisica teorica}} & {\ul {Laboratorio nucleare}}\\
            {\small \emph {Lagrangian formulation and Noether's theorem.}} & {\small \emph {Radiation interactions. Proportional counters,}}\\
            {\small \emph {Klein--Gordon, electromagnetic and Dirac fields.}} & {\small \emph {scintillation and semiconductor detectors.}}\\
            {\small D. Dominici} & {\small S. Barlini, A. Stefanini}\\
            {\small [6 \textsc{cfu}] 27/30, 2018-02-08} & {\small [9 \textsc{cfu}] 29/30, 2018-07-06}\\ [3mm]

            {\ul {Laboratorio subnucleare}} & {\ul {Machine learning}}\\
            {\small \emph {Stopping power and time of flight. Multi-wire}} & {\small \emph {Supervised-unsupervised learning. Linear models}}\\
            {\small \emph {proportional chamber and Cherenkov detectors.}} & {\small \emph {and support vector machines. Neural networks.}}\\
            {\small R. D'Alessandro, L. Bonechi, F. Bucci} & {\small P. Frasconi}\\
            {\small [9 \textsc{cfu}] 27/30, 2019-03-18} & {\small [6 \textsc{cfu}] 30/30, 2019-06-18}
        \end{tabular*}
    
    \vskip 12mm
    
    % CONFERENCES
    \section*{Conferences and Workshops}
        \begin{tabular*}{\textwidth}{rl @{\extracolsep{\fill}} r}
            {\small 2020} & \href{https://site.unibo.it/school-physical-sensing-and-processing/en}{\textbf{Summer School in Physical Sensing and Processing}} & {\small online}\\
            & \multicolumn{2}{l}{\small \emph{Summer School aimed to discuss the pipeline from the detection to the processing of physical signals,}}\\
            & \multicolumn{2}{l}{\small \emph {focusing on the big data complexity and the computational aspects.}}\\ [3 mm]
            
            {\small 2019} & \href{http://www.diitet.cnr.it/workshop-artificial-intelligence-in-health-and-well-being/}{\textbf{Artificial Intelligence in Health and Well-Being}} & {\small Firenze, Italy}\\
            & \multicolumn{2}{l}{\small \emph{Workshop on Artificial Intelligence applied to health and well-being.}}\\ [3 mm]
            
            {\small 2019} & \href{http://ai-sf.it/cisf19/}{\textbf{Italian Conference of Physics Students 2019}} & {\small Milano, Italy}\\
            & \multicolumn{2}{l}{\small \emph{Participation with a workshop titled ``Dietro l'Articolo SdR''.}}\\ [3 mm]
            
            {\small 2018} & \href{http://ai-sf.it/cisf18/}{\textbf{Italian Conference of Physics Students 2018}} & {\small Pisa, Italy}\\
            & \multicolumn{2}{l}{\small \emph{Representative for the Florence Local Committee of the Italian Association of Physics Student.}}\\ [3 mm]
            
            {\small 2017} & \href{http://www.ai-sf.it/cisf2017/}{\textbf{Italian Conference of Physics Students 2017}} & {\small Bari, Italy}\\
            & \multicolumn{2}{l}{\small \emph{Participation with a poster titled ``I ragazzi di via Panisperna''.}}\\ [3 mm]
        \end{tabular*}
        
    \pagebreak
    
%    % OTHER UNIVERSITY COURSES
%    \section*{Other university classes}
%        \begin{itemize}
%            \item \ul{Impresa Campus Unifi} \quad {\small [Sep. 2018 -- Feb. 2019]}\\
%            {\small \emph {Development of innovative ideas and business culture in university context.}}\\
%           {\small \emph {First phase: Team Building and business ideas focus.}}\\
%            {\small \emph {Second phase: training path and personal expert support.}}
%            
%            \item \ul{Palestra di intrapredenza} \quad {\small [March 2018]}\\
%            {\small \emph {Evaluation of personal motivation and entrepreneurial propensity.}}\\
%            {\small \emph {Training program planned with Design Thinking and Lego Serious Play.}}
%        \end{itemize}
        

    % PROFESSIONAL EXPERIENCE
    \section*{Professional experience}
        \begin{tabular*}{\textwidth}{rl @{\extracolsep{\fill}} r}
            {\small Sep. 2019 -- Dec. 2019} & \ul{INFN fellowship} & {\small Istituto Nazionale di Fisica Nucleare}\\
            & \multicolumn{2}{l}{\small \emph {Scholarship funded by the National Institute for Nuclear Physics (INFN) for}}\\
            & \multicolumn{2}{l}{\small \emph{spending three months at CERN.}}\\ [3 mm]
        
            {\small Nov. 2018 -- present} & \ul{Teaching assistant} & {\small Università degli Studi di Firenze}\\
            & \multicolumn{2}{l}{\small \emph {Support to teaching activities and Physics laboratories (with Python tutorials).}}\\ [3 mm]
            
            {\small June 2018 -- Oct. 2018} & \ul{Student assistant} & {\small Università degli Studi di Firenze}\\
            & \multicolumn{2}{l}{\small \emph {Welcome and orientation service for Science students.}}\\ [3 mm]
            
            {\small Sep. 2016 -- present} & \underline{Maths and Physics reps teacher}\\
            & \multicolumn{2}{l}{\small \emph {Private lessons for high-school and university students.}}\\ [3 mm]
            
            {\small Sep. 2016 -- Dec. 2016} & \ul{Science Library assistant} & {\small Università degli Studi di Firenze}\\
            & \multicolumn{2}{l}{\small \emph {Support to Science Library workers and book arrangement.}}\\ [3 mm]
            
            {\small March 2011 -- Aug. 2013} & \ul{Basketball officer} & {\small Federazione Italiana Pallacanestro}\\
            & \multicolumn{2}{l}{\small \emph {Referee assistant during basketball matches.}}\\
        \end{tabular*}
    
    \vskip 12mm
    
    % ITALIAN ASSOCIATION OF PHYSICS STUDENTS (AISF)
    \section*{Involvement in Italian Association of Physics Students (AISF)}
        \begin{tabular*}{\textwidth}{r @{\extracolsep{\fill}} l}
            {\small Nov. 2020 -- present} & \ul{Vice President of the Italian Association of Physics Students}\\ 
            & {\small \emph {Support to Association organization.}}\\ [3mm]
            
            {\small Oct. 2019 -- present} & \ul{Secretary of the Italian Association of Physics Students}\\
            & {\small \emph {Member of the Executive Committee that organizes and coordinates}}\\
            & {\small \emph {the Italian Local Committees.}}\\ [3mm]
                                            
            {\small June 2019} & {Participation at Caffé-Scienza  event}\\
            & {\small \emph {Presentation of the Florence Local Committee vulgarization activities.}}\\ [3mm]
        
            {\small May 2019} & {Organizer of the outreach event ``Tra clima e cocktail''}\\
            & {\small \emph {In collaboration with Italian Climate Network, CNR and University of Florence.}}\\ [3mm]
        
            {\small May 2019} & {Organizer of the outreach event ``Viaggio al Polo''}\\
            & {\small \emph {In collaboration with Caffè-Scienza, INFN and University of Florence.}}\\ [3mm]
                                            
            %{\small March 2019} & {Contributor at CISF19}\\
            %& {\small \emph {Workshop titled ``Dietro l'Articolo SdR''.}}\\ [3mm]
                                            
            %{\small March 2019} & {Representative for the Florence Local Committee at CISF19}\\
            %& {\small \emph {Presentation of Florence LC events at Italian Conference of Physics Students.}}\\ [3mm]
                                            
            {\small Nov. 2018 -- May 2019} & \ul{President of the Florence Local Committee}\\
            & {\small \emph {Organization of events and representative for about 40 members.}}\\ [3mm]
                                            
            {\small May 2018 -- present} & \ul{Member of the Editorial Board of \emph{Sistemi di Riferimento}}\\
            & {\small \emph {Vulgarization activity to valorise Italian Physics Departments.}}\\ [3mm]
                                            
            {\small Apr. 2018 -- May 2018} & {Organizer of the Florence International Day of Light 2018}\\
            & {\small \emph {Three Science Evenings focused on light at literary cafe La Cité.}}\\ [3mm]
                                            
            %{\small Apr. 2018} & {Representative for the Florence Local Committee at CISF18}\\
            %& {\small \emph {Presentation of Florence LC events at Italian Conference of Physics Students.}}\\ [3mm]
                                            
            {\small Dec. 2017 -- Nov. 2018} & \ul{Vice President of the Florence Local Committee}\\
            & {\small \emph {Support to events organization and Local Committee coordination.}}\\ [3mm]
                                            
            {\small Oct. 2017 -- Nov. 2017} & {Organizer for the Florence IAPS School Day}\\
            & {\small \emph {Lessons and experiments about acoustics to high-school students.}}\\ [3mm]
                                            
            {\small Oct. 2017} & {Organizer of the outreach event ``The new particles of LHCb''}\\
            & {\small \emph {In collaboration with the LHCb Florence Group.}}\\ [3mm]
                                            
            %{\small May 2017} & {Contributor at CISF17}\\
            %& {\small \emph {Poster titled ``I ragazzi di via Panisperna''.}}
        \end{tabular*}
    
    \pagebreak
    
    % OUTREACH AND VULGARIZATION
    \section*{Outreach and vulgarization}
        \begin{tabular*}{\textwidth}{rl @{\extracolsep{\fill}} r}
            {\small May 2018} & \ul{Jury member} & {\small Premio Galileo}\\
            & \multicolumn{2}{l}{\small \emph {Premio Galileo is a literary prize for science communication.}}\\ [3mm]
                               
            {\small Jan. 2017 -- Dec. 2017} & \ul{Blog author} & {\small Fisici Senza Palestra}\\
            & \multicolumn{2}{l}{\small \emph {Dissemination of science on \href{https://www.fisicisenzapalestra.com}{\textit{Fisici Senza Palestra}} blog.}}
        \end{tabular*}
        
   \vskip 9mm
    
    % TECHNICAL SKILLS
    \section*{Technical skills}
        \begin{tabular}{rll}
            {\small Operating systems:} & Windows\\
            & Mac OS\\
            & Linux\\ [3mm]
                                        
            {\small Programming languages:} 
            & Python & {\small \emph {with PyROOT, NumPy and Pandas}}\\
            & C/C++ & {\small \emph {with ROOT}}\\ [3mm]
            
            {\small Machine Learning frameworks:} 
            & Scikit-Learn\\
            & TensorFlow & {\small \emph {with Keras}}\\ [3 mm]
                                            
            {\small Other suites:} 
            & Google Workspace\\
            & Microsoft Office & {\small \emph {certified with ECDL (May 2012)}}\\
            & \LaTeX\\
            & Git
        \end{tabular}

    \vskip 9mm

    % LANGUAGES    
    \section*{Languages}
        \begin{tabular}{rl}
            Italian & {\small \emph {mother tongue}}\\
            English & {\small \emph {certificated as B1 (May 2013)}}\\
            Spanish & {\small \emph {certificated as B1 (July 2011)}}
        \end{tabular}

    \vskip 15mm

    \noindent
    {\small In compliance with the art. 46 and 47 of D.P.R. 445/2000 and further modifications, and aware of the consequences of making 
    false statements, falsehood of acts and use of false facts, punishable by the law according to art. 76 D.P.R. n. 445/2000
    and art. 496 of the Italian Penal Code, under my own responsibility, I declare that the information and facts in this document
    are truthful.}
  
    \vskip 6mm 
  
    \noindent
    {\small Firenze, \today}
    \hfill {\small Matteo Barbetti}\\ [3mm]
    \null \hfill \includegraphics[width=0.30\textwidth]{academic-cv/phd-applications/img/signature.png}
  
\end{document}