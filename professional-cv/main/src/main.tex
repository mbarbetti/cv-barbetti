% FortySecondsCV LaTeX template
% Copyright © 2019 René Wirnata <rene.wirnata@pandascience.net>
% Licensed under the 3-Clause BSD License. See LICENSE file for details.
%
% Attributions
% ------------
% * fortysecondscv is based on the twentysecondcv class by Carmine Spagnuolo 
%   (cspagnuolo@unisa.it), released under the MIT license and available under
%   https://github.com/spagnuolocarmine/TwentySecondsCurriculumVitae-LaTex
% * further attributions are indicated immediately before corresponding code


%-------------------------------------------------------------------------------
%                             ADDITIONAL PACKAGES
%-------------------------------------------------------------------------------
\documentclass[
  a4paper, 
%   showframes,
%   vline=2.2em,
%   maincolor=cvgreen,
%   sectioncolor=red,
%   subsectioncolor=orange,
%   itemtextcolor=black!80,
%   sidebarwidth=0.4\paperwidth,
%   topbottommargin=0.03\paperheight,
%   leftrightmargin=20pt,
%   proilepicsize=4.5cm,
]{fortysecondscv}

% improve word spacing and hyphenation
\usepackage{microtype}
\usepackage{ragged2e}

% take care of proper font encoding
\ifxetexorluatex
	\usepackage{fontspec}
	\defaultfontfeatures{Ligatures=TeX}
% \newfontfamily\headingfont[Path = fonts/]{segoeuib.ttf} % local font
\else
	\usepackage[utf8]{inputenc}
	\usepackage[T1]{fontenc}
% \usepackage[sfdefault]{noto} % use noto google font
\fi

% enable mathematical syntax for some symbols like \varnothing
\usepackage{amssymb}

% bubble diagram configuration
\usepackage{smartdiagram}
\smartdiagramset{
  % defaut font size is \large, so adjust to harmonize with sidebar layout
  bubble center node font = \footnotesize,
  bubble node font = \footnotesize,
  % default: 4cm/2.5cm; make minimum diameter relative to sidebar size
  bubble center node size = 0.4\sidebartextwidth,
  bubble node size = 0.25\sidebartextwidth,
  distance center/other bubbles = 1.5em,
  % set center bubble color
  bubble center node color = maincolor!70,
  % define the list of colors usable in the diagram
  set color list = {maincolor!10, maincolor!40,
  maincolor!20, maincolor!60, maincolor!35},
  % sets the opacity at which the bubbles are shown
  bubble fill opacity = 0.8,
}


%-------------------------------------------------------------------------------
%                            PERSONAL INFORMATION
%-------------------------------------------------------------------------------

%% mandatory information
% your name
\cvname{Matteo Barbetti}
% job title/career
\cvjobtitle{Ph.D. Student in\\[0.2em] 
            Smart Computing}

%% optional information
% profile picture
\cvprofilepic{professional-cv/main/img/profile.png}
% date of birth
\cvbirthday{May 6, 1994}
% short address/location, use \newline if more than 1 line is required
\cvaddress{Via Vittorio Emanuele II 193,
           \newline{50134, Firenze, Italy}}
% phone number
\cvphone{+39 347 8595002}
% email address
\cvmail{matteo.barbetti.94@gmail.com}
% personal website
\cvsite{https://mbarbetti.github.io}
% LinkedIn profile name
%\cvlinkedin{matteo-barbetti}
% GitHub profile name
%\cvgithub{mbarbetti}

%-------------------------------------------------------------------------------
%                              SIDEBAR 1st PAGE
%-------------------------------------------------------------------------------
% add more profile sections to sidebar on first page
\addtofrontsidebar{
	% include gosquare national flags from https://github.com/gosquared/flags;
	% naming according to ISO 3166-1 alpha-2 country codes
	\graphicspath{{professional-cv/img/flags/}}
	
	\vspace{0.5cm}
	
    \profilesection{About Me}
	\aboutme{
		Ph.D. Student in Smart Computing at the University of Florence, I deal with Artificial Intelligence and its applications to Particle and Medical Physics. Passionate about innovation and scientific dissemination, I never refuse going out to discuss new ideas.
	}
	
	\vspace{0.5cm}
	
	\profilesection{Memberships}
	\begin{memberships}
		\membership[4em]{professional-cv/main/img/AISF_logo.png}
		{\href{http://ai-sf.it}{Associazione Italiana
		                        \newline{Studenti di Fisica}}}\\
		\membership[4em]{professional-cv/main/img/INFN_logo.png}
		{\href{https://www.fi.infn.it}{Istituto Nazionale di
		                               \newline{Fisica Nucleare}}}\\
%    	\membership[4em]{professional-cv/main/img/CERN_logo.png}
%    	{\href{https://home.cern}{Conseil Européen pour 
%    	                          \newline{la Recherche Nucléaire}}}\\
		\membership[4em]{professional-cv/main/img/LHCb_logo.jpg}
		{\href{http://lhcb-public.web.cern.ch/lhcb-public/}{LHCb Collaboration}}
%		\membership[4em]{}{\rule{\linewidth}{1pt}}
	\end{memberships}
	
	\vspace{0.5cm}
	
	% social network accounts incl. proper hyperlinks
	\profilesection{Social Network}
		\begin{icontable}{2.5em}{1em}
			\social{\faLinkedinSquare}
				{https://www.linkedin.com/in/matteo-barbetti}
				{matteo-barbetti}
			\social{\faTwitter}
				{https://twitter.com/mbarbetz}
				{@mbarbetz}
			\social{\faGithub}
				{https://github.com/mbarbetti}
				{mbarbetti}
%			\social{\faGitlab}
%				{https://gitlab.com/mbarbetti}
%				{mbarbetti}
		\end{icontable}
}


%-------------------------------------------------------------------------------
%                              SIDEBAR 2nd PAGE
%-------------------------------------------------------------------------------
\addtobacksidebar{

	\vspace{0.75cm}

	\profilesection{Soft Skills}
        \skill{\faSitemap}{Complex problem-solving}
		\skill{\faRocket}{Analytical thinking and innovation}
		\skill{\faLineChart}{Critical thinking and analysis}
 		\skill{\faSearchPlus}{Attention to detail, trustworthiness}
 		\skill{\faUsers}{Leadership and social influence}

    \vspace{0.75cm}

	\profilesection{Technical Skills}
		\skill{\faLaptop}{\underline{Operating systems}}
		    \skill[1.8em]{\faWindows}{Windows}
		    \skill[1.8em]{\faApple}{Mac OS}
		    \skill[1.8em]{\faLinux}{Linux}
		\skill{\faTerminal}{\underline{Programming languages}}
		    \pointskill[1.8em]{\faCode}{Python}{4}[5]
		    \pointskill[1.8em]{\faCode}{C/C++}{2}[5]
%		    \pointskill[1.8em]{\faCode}{LabVIEW}{2}[5]
%		    \pointskill[1.8em]{\faCode}{Bash}{1}[5]
		\skill{\faCubes}{\underline{Machine Learning frameworks}}
		    \skill[1.8em]{\faCube}{Scikit-Learn}
		    \skill[1.8em]{\faCube}{TensorFlow}
		\skill{\faCogs}{\underline{Other suites}}
		    \skill[1.8em]{\faCog}{Visual Studio Code}	
		    \skill[1.8em]{\faCog}{Google Workspace}		 
		    \skill[1.8em]{\faCog}{Microsoft Office}
		    \skill[1.8em]{\faCog}{LaTeX}
		    \skill[1.8em]{\faCog}{Git}

    \vspace{0.75cm}

	\profilesection{Languages}
		\pointskill{\flag{professional-cv/main/img/flags/IT.png}}{Italian}{5}[5]
  	    \pointskill{\flag{professional-cv/main/img/flags/GB.png}}{English}{3}[5]
  	    \pointskill{\flag{professional-cv/main/img/flags/ES.png}}{Spanish}{1}[5]
  	    
%	\profilesection{Diagrams}
%	\chartlabel{Bubble Diagram}
%	\begin{figure}\centering
%		\smartdiagram[bubble diagram]{
%			\textcolor{white}{\textbf{Being a}} \\ 
%			\textcolor{white}{\textbf{Panda}}, % center bubble	
%			\textcolor{black!90}{Eating},
%			\textcolor{black!90}{Sleeping},
%			\textcolor{black!90}{Rolling},
%			\textcolor{black!90}{Playing},
%			\textcolor{black!90}{Chilling}
%		}
%	\end{figure}
%
%	\chartlabel{Wheel Chart}
%
%	\wheelchart{4em}{2em}{%
% 	20/3em/maincolor!50/Chill,
%	15/3em/maincolor!15/Play,
%  	30/4em/maincolor!40/Sleep,
%  	20/3em/maincolor!20/Eat
%	}
%
%	\profilesection{Barskills}
%	\barskill{\faSkyatlas}{Wearing asian rice hats}{60}
%	\barskill{\faImage}{Playing Chess}{30}
%	\barskill{\faMusic}{Playing the bamboo flute}{50}
}


%-------------------------------------------------------------------------------
%                         TABLE ENTRIES RIGHT COLUMN
%-------------------------------------------------------------------------------
\begin{document}

\makefrontsidebar


\cvsection{Education}

%\cvsubsection{Postgraduate Training}
%\begin{cvtable}[2]
%	\cvitem{2009 -- 2010}{Post-Doc Panda Studies}{Panda Academy}
%		{In-depth studies on the impact of bamboo nutrition for young pandas and
%		its relation to relaxing, sleeping and snoozing parts of the day.}
%	\cvitem{2008 -- 2009}{Research Stay Europe}{European Panda Labs}
%		{Spending one year abroad teaching european panda facilities about the
%		newest findings and research in the field of asian rice hat covers and
%		applications for bamboo as a material.}
%\end{cvtable}

\cvsubsection{Study}
\begin{cvtable}[2]
	\cvitem{2020 -- now}{\href{https://smartcomputing.unifi.it/}{Ph.D. in Smart Computing}}
	{Università degli Studi di Firenze}
		{\underline{Focus:} Multidisciplinary Ph.D. program aimed to provide expertise on advanced computational methods to meet the needs of society.\\
		\underline{Topic:} Smart Computing Techniques applied to Medical Physics, Nuclear Physics and Particle Physics.}
	\cvitem{2017 -- 2020}{Master's degree in Particle Physics}
	{Università degli Studi di Firenze}
		{\underline{Focus:} Curriculum of studies oriented to Physics Data Analysis with a strong focus on modern Computing Technologies, like Machine Learning and Artificial Intelligence.\\
		\underline{Score:} $110/110$ \emph{cum laude}}
	\cvitem{}{Master thesis}{Genève, Switzerland}
		{\underline{Focus:} Development of a Python-based simulation framework for Particle Physics applications integrated with Deep Learning solutions to parameterize the detector response of the LHCb experiment.\\
		\underline{Title:} ``Techniques for parametric simulation with deep neural networks and implementation for the LHCb experiment at CERN and its future upgrades''.}
	\cvitem{2013 -- 2017}{Bachelor's degree in Physics}
	{Università degli Studi di Firenze}
		{\underline{Focus:} Standard course of study in Physics and Astrophysics during which I discovered a natural inclination towards Experimental Science and Data Analysis.\\
		\underline{Score:} $110/110$}
	\cvitem{}{Bachelor thesis}{Genève, Switzerland}
		{\underline{Focus:} Data Analysis work in Particle Physics in collaboration with the 
		 LHCb experiment at CERN.\\
		 \underline{Title:} ``Study of the charmonium resonances in $B^+ \to p \bar{p} K^+$ 
		 and $B^+ \to p \bar{p} \gamma K^+$ decays with the LHCb experiment at CERN''.}
	\cvitem{2008 -- 2013}{Previous education}{Liceo Scientifico G. Galilei di Terni}
		{\underline{Focus:} First general smattering on Scientific Approach and Critical Thinking thanks to which I decided to continue studies in Physics.\\
		 \underline{Score:} $100/100$ \textit{cum laude}}
\end{cvtable}

\vskip 2.5mm

\cvsubsection{Other courses}
\begin{cvtable}[2]
	\cvitem{2018 -- 2019}{\href{https://www.unifi.it/impresacampus}{Impresa Campus Unifi}}{Università degli Studi di Firenze}
		{\underline{Focus:} Development of innovative ideas and business culture in university context   with Team Building path and personal expert support.}
	\cvitem{2018}{\href{https://www.unifi.it/vp-8678-palestra-di-intraprendenza.html}{Palestra di Intrapredenza}}{Università degli Studi di Firenze}
		{\underline{Focus:} Evaluation of personal motivation and entrepreneurial propensity thanks to Design Thinking and LEGO Serious Play\textsuperscript{\textregistered} methods.}
\end{cvtable}

\vskip 5mm

\cvsection{Experience}
\cvsubsection{Tutoring}
\begin{cvtable}[2]
	\cvitem{2018 -- now}{Teaching assistant}{Università degli Studi di Firenze}
	    {Support to laboratory activities for the BSc degree course in Physics. Since October 2020, I am responsible for brief lessons of Python for the third year students.}
%	\cvitem{2018}{Student assistant}{Università degli Studi di Firenze}
%	    {Welcome and orientation service for science students.}
	\cvitem{2016 -- now}{Maths and Physics Reps Teacher}{}
	    {Private lessons for high-school and university students.}
\end{cvtable}

\newpage
\makebacksidebar

\cvsubsection{Italian Association of Physics Student (AISF)}
\begin{cvtable}[2]
    \cvitem{2020 -- now}{Vice President}{}
	    {Support to Association organization.}
	\cvitem{2019 -- now}{Secretary}{}
	    {Member of the Executive Committee that organizes and coordinates the Italian Local Committees.}
	\cvitem{2018 -- 2019}{President of the Florence Local Committee}{}
	    {Organization of events and representative for about 40 members.}
	\cvitem{2018 -- now}{Member of the Editorial Board of \textit{Sistemi di Riferimento}}{}
	    {Vulgarization activity to valorise Italian Physics Departments.}
	\cvitem{2017 -- 2018}{Vice President of the Florence Local Committee}{}
	    {Support to events organization and Local Committee coordination.}
\end{cvtable}

%\newpage
%\makebacksidebar

%\cvsubsection{Other experiences}
%\begin{cvtable}[2]
%	\cvitem{2016}{Science Library assistant}{Università degli Studi di Firenze}
%	    {Support to science library workers and book arrangement.}
%	\cvitem{2011 -- 2013}{Basketball officer}{Federazione Italiana Pallacanestro}
%	    {Referee assistant during basketball matches.}
%\end{cvtable}


\cvsection{Conferences, Workshops and Schools}
\begin{cvtable}[2]
    \cvitem{2021}{\href{https://indico.cern.ch/event/1017080/}{8th Thematic CERN School of Computing}}{online}
        {\underline{Focus:} Scientific software for heterogeneous architectures.}
%    \cvitem{2021}{\href{https://agenda.infn.it/event/25855/}{First ML-INFN Hackathon}}{online}
 %       {\underline{Focus:} Introduction to Machine Learning topics interesting for INFN (use-case selected: Medical Physics).}
    \cvitem{2021}{\href{https://agenda.infn.it/event/25889/}{Workshop della Commissione Calcolo e Reti dell'INFN}}{online}
        {Participation with a presentation titled ``Simulating the LHCb detector with Generative Adversarial Networks''.}
    \cvitem{2021}{1st CloudBank EU Workshop}{online}
        {Participation with a presentation titled ``LHCb deployment in AWS''.}
    \cvitem{2020}{\href{https://site.unibo.it/school-physical-sensing-and-processing/en}{Summer School in Physical Sensing and Processing}}{online}
	    {\underline{Focus:} Sensing devices, DAQ systems and data processing strategies across various Physics subdomains.}
	\cvitem{2019}{\href{http://www.diitet.cnr.it/workshop-artificial-intelligence-in-health-and-well-being/}{Artificial Intelligence in Health and Well-Being}}{Firenze, Italy}
	    {\underline{Focus:} Artificial Intelligence applied to health and well-being.}
%	\cvitem{2019}{\href{http://ai-sf.it/cisf19/}{Italian Conference of Physics Students 2019}}{Milano, Italy}
%	    {Participation with a workshop titled ``\textit{Dietro l'Articolo SdR}''.}
%	\cvitem{2018}{\href{http://ai-sf.it/cisf18/}{Italian Conference of Physics Students 2018}}{Pisa, Italy}
%	    {Representative for the Florence Local Committee.}
%	\cvitem{2017}{\href{http://www.ai-sf.it/cisf2017/}{Italian Conference of Physics Students 2017}}{Bari, Italy}
%	    {Participation with a poster titled ``\textit{I ragazzi di via Panisperna}''.}
\end{cvtable}


\cvsection{Outreach and Dissemination}
\begin{cvtable}[2]
    \cvitem{2021}{Book author}{Sassi Junior}
        {On behalf of the INFN, I have prepared a paragraph dedicated to Artificial Intelligence for ``Inventions'', a science book for kids.}
	\cvitem{2018}{Jury member}{Premio Galileo}
	    {\textit{Premio Galileo} is a literary prize for science communication.}
	\cvitem{2017 -- 2018}{Blog author}{Fisici Senza Palestra}
	    {Dissemination of science on \href{https://www.fisicisenzapalestra.com}{\textit{Fisici Senza Palestra}} blog.}
\end{cvtable}


\cvsection{Awards}
\begin{cvtable}[2]
	\cvitem{2019}{INFN scolarship}{Istituto Nazionale di Fisica Nucleare}
	    {Scholarship funded by the National Institute for Nuclear Physics (INFN) for spending three months at CERN.}
	\cvitem{2017}{Scholarship for thesis abroad}{Università degli Studi di Firenze}
	    {Sholarship funded by the University of Florence for spending a part of the bachelor thesis work at CERN.}
	\cvitem{2013}{Scholar merit award}{Liceo Scientifico G. Galilei di Terni}
	    {Merit award for the excellence of high-school career.}
\end{cvtable}


%\cvsection{Publications}
%\begin{cvtable}
%	\cvpubitem{Cooking: 100 recipes for lazy Pandas}{Me and My Panda Friends}
%		{Panda's Culinary World}{2010}
%	\cvpubitem{Pandastasia}{Still Me}{Bamboo Books Assoc.}{2005}
%\end{cvtable}


%\cvsection{Extra-Curricular Activities}
%\begin{cvtable}
%	\cvitemshort{Relaxing}{Master the fine art of relaxing everywhere}
%	\cvitemshort{Music}{Playing the bamboo flute in the 1st Panda Orchestra}
%	\cvitemshort{Education}{Teaching young pandas to be more panda-like}
%\end{cvtable}


%\cvsection{section}
%\cvsubsection{Subsection}
%\begin{cvtable}
%	\cvitem{<dates>}{<cv-item title>}{<location>}{<optional: description>}
%\end{cvtable}


%\cvsubsection{One-line without description}
%\begin{cvtable}
%	\cvitem{Award}{One-line description}{Sponsor}{}
%	\cvitem{Award}{One-line description}{Sponsor}{}
%	\cvitem{Award}{One-line description}{Sponsor}{}
%\end{cvtable}

%\cvsection{cvitemshort}
%\cvsubsection{One-line}
%\begin{cvtable}
%	\cvitemshort{Key}{Some further description}
%	\cvitemshort{Key}{Some further description}
%	\cvitemshort{Key}{Some further description}
%\end{cvtable}

%\cvsubsection{Multi-line with longer description}
%\begin{cvtable}
%	\cvitemshort{Key}{Some further description. Can fill even more than
%		only one single line while still keeping the correct indendation level.}
%	\cvitemshort{Key}{Some further description. Can fill even more than
%		only one single line while still keeping the correct indendation level.}
%	\cvitemshort{Key}{Some further description. Can fill even more than
%		only one single line while still keeping the correct indendation level.}
%\end{cvtable}

%\cvsection{cvpubitem}
%\begin{cvtable}
%	\cvpubitem{Publication title}{Authors}{Journal}{Year}
%	\cvpubitem{Publication title}{Authors}{Journal}{Year}
%	\cvpubitem{Publication title that is spanning over multiple lines and still
%		does not look too bad}{Authors}{Journal}{Year}
%\end{cvtable}

\cvsignature

\end{document} 
