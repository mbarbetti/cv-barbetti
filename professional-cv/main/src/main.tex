
%-------------------------------------------------------------------------------
%                             ADDITIONAL PACKAGES
%-------------------------------------------------------------------------------

\documentclass[
  a4paper, 
%   showframes,
%   vline=2.2em,
%   maincolor=cvgreen,
%   sectioncolor=red,
%   subsectioncolor=orange,
%   itemtextcolor=black!80,
%   sidebarwidth=0.4\paperwidth,
%   topbottommargin=0.03\paperheight,
%   leftrightmargin=20pt,
%   proilepicsize=4.5cm,
]{fortysecondscv}

% improve word spacing and hyphenation
\usepackage{microtype}
\usepackage{ragged2e}

% take care of proper font encoding
\ifxetexorluatex
	\usepackage{fontspec}
	\defaultfontfeatures{Ligatures=TeX}
% \newfontfamily\headingfont[Path = fonts/]{segoeuib.ttf} % local font
\else
	\usepackage[utf8]{inputenc}
	\usepackage[T1]{fontenc}
% \usepackage[sfdefault]{noto} % use noto google font
\fi

% enable mathematical syntax for some symbols like \varnothing
\usepackage{amssymb}

% bubble diagram configuration
\usepackage{smartdiagram}
\smartdiagramset{
  % defaut font size is \large, so adjust to harmonize with sidebar layout
  bubble center node font = \footnotesize,
  bubble node font = \footnotesize,
  % default: 4cm/2.5cm; make minimum diameter relative to sidebar size
  bubble center node size = 0.4\sidebartextwidth,
  bubble node size = 0.25\sidebartextwidth,
  distance center/other bubbles = 1.5em,
  % set center bubble color
  bubble center node color = maincolor!70,
  % define the list of colors usable in the diagram
  set color list = {maincolor!10, maincolor!40,
  maincolor!20, maincolor!60, maincolor!35},
  % sets the opacity at which the bubbles are shown
  bubble fill opacity = 0.8,
}

% enable alternative underlining
\usepackage{soul}

% enable using special symbol
\usepackage{textcomp}

% custom packages
\usepackage{titlesec}
\usepackage{dashrule}


%-------------------------------------------------------------------------------
%                            PERSONAL INFORMATION
%-------------------------------------------------------------------------------

%% mandatory information
% your name
\cvname{Matteo Barbetti}
% job title/career
\cvjobtitle{Data Scientist}

%% optional information
% profile picture
\cvprofilepic{professional-cv/main/img/profile.png}
% date of birth
\cvbirthday{May 6, 1994}
% short address/location, use \newline if more than 1 line is required
%\cvaddress{Via Reginaldo Giuliani 86,
%           \newline{50141, Firenze, Italy}}
% phone number
\cvphone{+39 3478595002}
% email address
\cvmail{matteo.barbetti.94@gmail.com}
% personal website
%\cvsite{https://mbarbetti.github.io}
% LinkedIn profile name
\cvlinkedin{matteo-barbetti}
% GitHub profile name
\cvgithub{mbarbetti}


%-------------------------------------------------------------------------------
%                              SIDEBAR 1st PAGE
%-------------------------------------------------------------------------------

% add more profile sections to sidebar on first page

\addtofrontsidebar{
	
	\vspace{0.8em}

    % +-------------+
    % |   SUMMARY   |
    % +-------------+
    \profilesection{Summary}
    \vspace{-0.5em}
	    %
        \longskill{\faPython}
          {\textbf{8+ years of Python development}, applied to scientific computing and industrial solutions.}
        %
        \vspace{0.5em}
        %
        \longskill{\faBrain}
          {\textbf{5+ years of hands-on ML experience}, designing and deploying predictive models.}
        %
        \vspace{0.5em}
        %
        \longskill{\faAtom}
          {\textbf{Former academic researcher in Particle Physics}, with strong foundations in data analysis and simulation.}
        %
        \vspace{0.5em}
        %
        \longskill{\faChartLine}
          {\textbf{AI consultant}, building data-driven solutions for real-world industrial problems.}
        %
        \vspace{0.5em}
        %
        \longskill{\faLightbulb}
          {\textbf{Strong interest in emerging technologies}, particularly AI, GenAI, and Quantum Computing.}

    \vspace{0.8em}

    % +---------------+
    % |   STRENGTHS   |
    % +---------------+
    \profilesection{Strengths}
    \vspace{-0.5em}
	    %
        \longskill{\faPuzzlePiece}
          {\textbf{Analytical problem solver.} Structured, data-driven approach to complex challenges.}
        %
        \vspace{0.5em}
        %
        \longskill{\faRocket}
          {\textbf{Autonomous, execution-driven.} Owns tasks end-to-end and delivers effective solutions.}
        %
        \vspace{0.5em}
        %
        \longskill{\faHandshake}
          {\textbf{Collaborative, client-focused.} Strong teamwork and clear communication aligned with client needs.}

}


%-------------------------------------------------------------------------------
%                              SIDEBAR 2nd PAGE
%-------------------------------------------------------------------------------

\addtobacksidebar{

    \vspace{1em}

    % +------------+
    % |   SKILLS   |
    % +------------+
	\profilesection{Technical Skills}
    \vspace{-0.5em}
        %
        \skill{\faLaptopCode}{\textbf{Data Science Toolkit}}
            %
            \skill[1.5em]{\faAngleRight}{
                \underline{Data handling:}\\ \texttt{numpy}, \texttt{pandas}, \texttt{geopandas}
            }
            %
            \skill[1.5em]{\faAngleRight}{
                \underline{Data visualization:}\\ \texttt{matplotlib}, \texttt{seaborn}
            }
            %
            \skill[1.5em]{\faAngleRight}{
                \underline{Classical ML models:}\\ \texttt{scikit-learn}, \texttt{xgboost}
            }
            %
            \skill[1.5em]{\faAngleRight}{
                \underline{Neural networks:}\\ \texttt{pytorch}, \texttt{tensorflow}, \texttt{keras}
            }
            %
            \skill[1.5em]{\faAngleRight}{
                \underline{Model optimization:}\\ \texttt{optuna}, \texttt{gurobipy}, \texttt{ortools}
            }
        %
        \vspace{0.5em}
        %
        \skill{\faServer}{\textbf{Scalable Computing}}
            %
            \skill[1.5em]{\faAngleRight}{
                \underline{Cloud Computing:} Azure \& AWS (compute, storage, ML services)
            }
            %
            \skill[1.5em]{\faAngleRight}{
                \underline{High-Performance Computing:} SLURM-based batch systems
            }
            %
            \skill[1.5em]{\faAngleRight}{
                \underline{GPU-powered acceleration:} CUDA-enabled workflows
            }
        %
        \vspace{0.5em}
        %
        \skill{\faBoxes}{\textbf{MLOps \& Deployment}}
            %
            \skill[1.5em]{\faAngleRight}{
                \underline{Containerization:}\\ Docker, Singularity / Apptainer
            }
            %
            \skill[1.5em]{\faAngleRight}{
                \underline{Orchestration:} Kubernetes
            }
            %
            \skill[1.5em]{\faAngleRight}{
                \underline{CI/CD pipelines:}\\ GitHub Actions, GitLab CI
            }

    \vspace{1em}

    % +-----------------+
    % |   MEMBERSHIPS   |
    % +-----------------+
	\profilesection{Memberships}
    \vspace{-1em}
        %
%        \simplemembership{Qiskit Advocate Program}
%          {2026 -- now}{\faUsers \, Community}
        %
%        [1.5em]
        %
        \simplemembership{Italian National Research Center for HPC, Big Data and Quantum Computing (ICSC)}
          {2023 -- 2024}{\faFlask \, Research}
        %
        [1.5em]
        %
        \simplemembership{LHCb Collaboration at CERN}
          {2017 -- 2024}{\faFlask \, Research}
        %
        [1.5em]
        %
        \simplemembership{Italian National Institute for Nuclear Physics (INFN)}
          {2017 -- 2024}{\faFlask \, Research}
        %
        [1.5em]
        %
        \simplemembership{Italian Association of Physics Students (AISF)}
          {2017 -- 2021}{\faUsers \, Community}

    \vspace{1em}

    % +---------------+
    % |   LANGUAGES   |
    % +---------------+
	\profilesection{Languages}
    % include gosquare national flags from https://github.com/gosquared/flags;
	% naming according to ISO 3166-1 alpha-2 country codes
        %
		\pointskill{
            \flag{professional-cv/main/img/flags/IT.png}
        }{Italian}{5}[5]
        %
  	    \pointskill{
            \flag{professional-cv/main/img/flags/GB.png}
        }{English}{3}[5]
        %
%  	    \pointskill{
%            \flag{professional-cv/main/img/flags/ES.png}
%        }{Spanish}{1}[5]


    % +--------------+
    % |   DIAGRAMS   |
    % +--------------+
%	\profilesection{Diagrams}
%	\chartlabel{Bubble Diagram}
%	\begin{figure}\centering
%		\smartdiagram[bubble diagram]{
%			\textcolor{white}{\textbf{Being a}} \\ 
%			\textcolor{white}{\textbf{Panda}}, % center bubble	
%			\textcolor{black!90}{Eating},
%			\textcolor{black!90}{Sleeping},
%			\textcolor{black!90}{Rolling},
%			\textcolor{black!90}{Playing},
%			\textcolor{black!90}{Chilling}
%		}
%	\end{figure}
%
%	\chartlabel{Wheel Chart}
%
%	\wheelchart{4em}{2em}{%
% 	    20/3em/maincolor!50/Chill,
%	    15/3em/maincolor!15/Play,
%  	    30/4em/maincolor!40/Sleep,
%  	    20/3em/maincolor!20/Eat
%	}


    % +---------------+
    % |   BARSKILLS   |
    % +---------------+
%	\profilesection{Barskills}
%	\barskill{\faSkyatlas}{Wearing asian rice hats}{60}
%	\barskill{\faImage}{Playing Chess}{30}
%	\barskill{\faMusic}{Playing the bamboo flute}{50}
    
}

%-------------------------------------------------------------------------------
%                         TABLE ENTRIES RIGHT COLUMN
%-------------------------------------------------------------------------------

\begin{document}

\makefrontsidebar

% +--------------+
% |   ABOUT ME   |
% +--------------+
%\cvsection{About Me}
%\vspace{1em}

%I am a \emph{Data Scientist} at the AI and GenAI Center of Excellence (CoE) founded by Accenture and Ammagamma in Modena, where I work on Artificial Intelligence projects with industrial applications for clients in the Utilities, Resources \& Chemicals sector. I transitioned from academia bringing a strong background in scientific computing, with a particular focus on scalable systems, HPC, and cloud-based infrastructures, combined with solid experience in data-driven and AI model development.\\
%
% In July 2024, I obtained a PhD with honors in Smart Computing from the University of Firenze. My research focused on Deep Generative Models (in particular GANs) for ultra-fast simulation in the LHCb experiment at CERN, strengthening my expertise in large-scale modeling and performance-oriented AI.\\
%
%Alongside my technical work, I am strongly interested in science communication and innovation-driven environments. I value open discussion and collaboration across diverse perspectives as key enablers for impactful and sustainable technological solutions.

%\vspace{1em}

% +----------------+
% |   EXPERIENCE   |
% +----------------+
\cvsection{Experience}
\vspace{-0.5em}

\begin{cvtable}[2]
    %
    \cvexpitem{Data Scientist}
      {Accenture}
      {2024 -- Present}{Modena (IT)}
      {AI consulting role in the Utilities, Resources \& Chemicals sector, developing data-driven solutions to optimize industrial systems and decision-making processes:}
        %
        \skill{\faAngleRight}{Led the design and industrialization of \textbf{data-driven optimizers for oil and gas applications}, integrating AI-based surrogate models to calibrate and optimize production systems using field data}\\
        %
        \skill{\faAngleRight}{Led multidisciplinary teams of Data Scientists and Software Engineers to develop and deploy \textbf{AI solutions within custom web applications} for industrial clients}\\
        %
        \skill{\faAngleRight}{Explored the \textbf{scalability of industrial optimization problems} on HPC infrastructures and emerging Quantum Computing paradigms}\\
        %
%        \skill{\faAngleRight}{Designed and implemented a \textbf{simulation orchestrator} to efficiently sample and generate data for training and improving AI-based surrogate models of gas storage systems}\\
        %
        \skill{\faAngleRight}{Delivered \textbf{technical training on Anomaly Detection} to client Data Science teams in the materials industry}\\
        %
        \skill{\faAngleRight}{Contributed to the design of \textbf{AI and GenAI proposals} for client tenders}\\
        %
    \vspace{-1em}
    \raisebox{0.8em}{%
        \hdashrule{\linewidth}{0.1pt}{2pt 2pt}%
    }\\ [-1em]
    %
    \cvexpitem{Technological Researcher}
      {INFN-CNAF}
      {2023 -- 2024}{Bologna (IT)}
      {Research position within the ICSC project, contributing to the operation of large-scale cloud and data center services:}
        %
        \skill{\faAngleRight}{Extended the LHCb fast simulation framework to run on \textbf{federated and heterogeneous computing infrastructures}}\\
        %
        \skill{\faAngleRight}{Served as the data center technical contact for LHCb, monitoring the use of \textbf{large-scale storage and computing resources} for simulation workflows}\\
        %
        \skill{\faAngleRight}{Designed and implemented a \textbf{GenAI-powered chatbot} to enhance user support services at the data center}\\
        %
    \vspace{-1em}
    \raisebox{0.8em}{%
        \hdashrule{\linewidth}{0.1pt}{2pt 2pt}%
    }\\ [-1em]
    %
    \cvexpitem{Doctoral Student}
      {University of Firenze} 
	  {2020 -- 2024}{Firenze (IT)}
      {Ph.D. at the intersection of Particle Physics and Computer Science, applying AI and scalable computing to scientific simulation:}
        %
        \skill{\faAngleRight}{Designed a cloud-based microservice orchestrating \textbf{large-scale optimization workflows} across on-premise, cloud, and HPC resources}\\
        %
        \skill{\faAngleRight}{Developed and optimized \textbf{GAN-based models for fast simulation} of the LHCb detector response using heterogeneous infrastructures}\\
        %
        \skill{\faAngleRight}{Integrated the optimized models into a pure Python simulation framework, achieving up to $\times 1000$ speed-up and \textbf{accelerating Physics analyses}}\\
        %
    \vspace{-1em}
    \raisebox{0.8em}{%
        \hdashrule{\linewidth}{0.1pt}{2pt 2pt}%
    }\\ [-1em]
    %
%	\cvexpitem{Senior Tutor}
%      {University of Firenze} 
%	  {2020 -- 2023}{Firenze (IT)}
%      {Scholarship funded by a local grant from the University of Firenze for a total of 150 hours/year to give introductory lessons on Python, discuss Physics exercises and support Physics Laboratory activities}
    %
%    \vspace{-1em}
%    \raisebox{0.8em}{%
%        \hdashrule{\linewidth}{0.1pt}{2pt 2pt}%
%    }\\ [-1em]
    %
%    \cvexpitem{Intern}
%      {INFN}
%      {2020}{Firenze (IT)}
%      {Curricular internship at the INFN Florence Division for a total of 150 hours dedicated to software development and AI models validation}
	%
%    \vspace{-1em}
%    \raisebox{0.8em}{%
%        \hdashrule{\linewidth}{0.1pt}{2pt 2pt}%
%    }\\ [-1em]
    %
%	\cvexpitem{Tutor}
%      {University of Firenze}
%      {2018 -- 2020}{Firenze (IT)}
%      {Scholarship funded by a local grant from the University of Firenze for a total of 200 hours/year to give introductory lessons on Python, discuss Physics exercises and support Physics Laboratory activities}
	%
%    \vspace{-1em}
%    \raisebox{0.8em}{%
%        \hdashrule{\linewidth}{0.1pt}{2pt 2pt}%
%    }\\ [-1em]
    %
    \cvexpitem{Research Intern}
      {CERN}
      {2019}{Geneva (CH)}
      {Selected for a three-month research internship at CERN to explore the use of \textbf{Deep Generative Models} (in particular GANs) for \textbf{fast simulation} within the LHCb experiment, marking my first hands-on experience with AI-driven approaches in large-scale scientific computing}
    %
%    \vspace{-1em}
%    \raisebox{0.8em}{%
%        \hdashrule{\linewidth}{0.1pt}{2pt 2pt}%
%    }\\ [-1em]
    %
%    \cvexpitem{Junior Tutor}
%      {University of Firenze}
%      {2018}{Firenze (IT)}
%      {Scholarship funded by a local grant from the University of Firenze for a total of 200 hours to take part to scientific dissemination event and support new students with orientation service}
    %
%    \vspace{-1em}
%    \raisebox{0.8em}{%
%        \hdashrule{\linewidth}{0.1pt}{2pt 2pt}%
%    }\\ [-1em]
    %
%    \cvexpitem{Intern}
%      {CERN}
%      {2017}{Geneva (CH)}
%      {Internship of two weeks at CERN funded by a local grant from the University of Firenze to finalize the bachelor thesis}
    %
\end{cvtable}

\newpage
\makebacksidebar

% +---------------+
% |   EDUCATION   |
% +---------------+
\cvsection{Education}
\vspace{-0.5em}

\begin{cvtable}[2]
    %
	\cveduitem{Ph.D. in Smart Computing}
      {University of Firenze}
      {2020 -- 2024}{Firenze (IT)}{Ph.D. entitled \emph{cum laude}}
      {\ul{Thesis:} Development of a scalable fast simulation framework for the LHCb experiment at CERN, based on Deep Generative Models and large-scale training on HPC and Cloud systems}
    %
    \vspace{-1em}
    \raisebox{0.8em}{%
        \hdashrule{\linewidth}{0.1pt}{2pt 2pt}%
    }\\ [-1em]
    %
	\cveduitem{M.Sc. in Particle Physics}
      {University of Firenze}
      {2017 -- 2020}{Firenze (IT)}{$110/110$ \emph{cum laude}}
      {\ul{Thesis:} Design of a Python-based simulation framework using AI surrogate models for fast detector simulation in the LHCb experiment at CERN}
	%
    \vspace{-1em}
    \raisebox{0.8em}{%
        \hdashrule{\linewidth}{0.1pt}{2pt 2pt}%
    }\\ [-1em]
    %
	\cveduitem{B.Sc. in Physics and Astrophysics}
      {University of Firenze}
      {2013 -- 2017}{Firenze (IT)}{$110/110$}
      {\ul{Thesis:} Data analysis and validation of simulated Particle Physics datasets from the LHCb experiment at CERN}
	%
\end{cvtable}

\vspace{1em}

% +--------------+
% |   LICENSES   |
% +--------------+
\cvsection{Licenses \& Certifications}
\vspace{-0.5em}

\begin{cvtable}[2]
    %
%    \cvcertitem{Quantum Computation using Qiskit v2.X Developer - Associate}
%      {IBM}{\faClipboardCheck \, Certification}{2026}
    %
%    \vspace{-1em}
%    \raisebox{0.8em}{%
%        \hdashrule{\linewidth}{0.1pt}{2pt 2pt}%
%    }\\ [-1em]
    %
%    \cvcertitem{Azure AI Engineer Associate (AI-102)}
%      {Microsoft}{\faClipboardCheck \, Certification}{2026}
    %
%    \vspace{-1em}
%    \raisebox{0.8em}{%
%        \hdashrule{\linewidth}{0.1pt}{2pt 2pt}%
%    }\\ [-1em]
    %
%    \cvcertitem{Azure AI Fundamentals (AI-900)}
%      {Microsoft}{\faClipboardCheck \, Certification}{2026}
    %
%    \vspace{-1em}
%    \raisebox{0.8em}{%
%        \hdashrule{\linewidth}{0.1pt}{2pt 2pt}%
%    }\\ [-1em]
    %
%    \cvcertitem{Quantum Chemistry with VQE}
%      {IBM}{\faAward \, Badge}{???}
    %
%    \vspace{-1em}
%    \raisebox{0.8em}{%
%        \hdashrule{\linewidth}{0.1pt}{2pt 2pt}%
%    }\\ [-1em]
    %
%    \cvcertitem{Variational Algorithm Design}
%      {IBM}{\faAward \, Badge}{???}
    %
%    \vspace{-1em}
%    \raisebox{0.8em}{%
%        \hdashrule{\linewidth}{0.1pt}{2pt 2pt}%
%    }\\ [-1em]
    %
    \cvcertitem{Quantum Machine Learning}
      {IBM}{\faAward \, Badge}{2026}
    %
    \vspace{-1em}
    \raisebox{0.8em}{%
        \hdashrule{\linewidth}{0.1pt}{2pt 2pt}%
    }\\ [-1em]
    %
%    \cvcertitem{Practical Introduction to Quantum-Safe Cryptography}
%      {IBM}{\faAward \, Badge}{???}
    %
%    \vspace{-1em}
%    \raisebox{0.8em}{%
%        \hdashrule{\linewidth}{0.1pt}{2pt 2pt}%
%    }\\ [-1em]
    %
%    \cvcertitem{Quantum Business Foundations}
%      {IBM}{\faAward \, Badge}{???}
    %
%    \vspace{-1em}
%    \raisebox{0.8em}{%
%        \hdashrule{\linewidth}{0.1pt}{2pt 2pt}%
%    }\\ [-1em]
    %
%    \cvcertitem{Quantum Diagonalization Algorithms}
%      {IBM}{\faAward \, Badge}{???}
    %
%    \vspace{-1em}
%    \raisebox{0.8em}{%
%        \hdashrule{\linewidth}{0.1pt}{2pt 2pt}%
%    }\\ [-1em]
    %
%    \cvcertitem{General Formulation of Quantum Information}
%      {IBM}{\faAward \, Badge}{???}
    %
%    \vspace{-1em}
%    \raisebox{0.8em}{%
%        \hdashrule{\linewidth}{0.1pt}{2pt 2pt}%
%    }\\ [-1em]
    %
%    \cvcertitem{Fundamentals of Quantum Algorithms}
%      {IBM}{\faAward \, Badge}{2026}
    %
%    \vspace{-1em}
%    \raisebox{0.8em}{%
%        \hdashrule{\linewidth}{0.1pt}{2pt 2pt}%
%    }\\ [-1em]
    %
    \cvcertitem{Basics of Quantum Information}
      {IBM}{\faAward \, Badge}{2025}
    %
\end{cvtable}

%\vspace{1em}

% +-----------------+
% |   CONFERENCES   |
% +-----------------+
%\cvsection{Conferences \& Workshops}
%\vspace{-0.5em}

%\begin{cvtable}[2]
    %
%    \cvconfitem{42nd ICHEP conference}
%      {2024}{Prague (CZ)}
%      {Participation with a talk on a flash simulation framework for Particle Physics based on Generative AI during an international conference}
    %
%    \cvconfitem{Workshop on computing at INFN}
%      {2024}{Palau (IT)}
%      {Participation with a talk on a NLP GenAI solution for enhancing the CNAF User Support unit during  a national workshop}
    %
%    \cvconfitem{1st EuCAIF conference}
%      {2024}{Amsterdam (NL)}
%      {Participation with a poster on a GenAI-based simulation framework for Particle Physics during an international conference}
    %
%    \cvconfitem{Fifth ML-INFN Hackathon: Advanced Level}
%      {2023}{Pisa (IT)}
%      {Tutorial preparation for an educational event on Machine Learning organized by INFN}
    %
%    \cvconfitem{26th CHEP conference}
%      {2023}{Norfolk, Virginia (USA)}
%      {Participation with a talk on a GenAI-based simulation framework for Particle Physics during an international conference}
    %
%    \cvconfitem{Third ML-INFN Hackathon: Advanced Level}
%      {2022}{Bari (IT)}
%      {Teaching activity during an educational event on Machine Learning organized by INFN}
    %
%    \cvconfitem{21st ACAT workshop}
%      {2022}{Bari (IT)}
%      {Participation with two posters on Machine Learning applications to Particle Physics during an international workshop}
    %
%    \cvconfitem{108\textdegree\ Congresso Nazionale della SIF}
%      {2022}{Milan (IT)}
%      {Participation with a talk on a Machine Learning application to Medical Physics to a national conference}
    %
%    \cvconfitem{4th European Congress of Medical Physics}
%      {2022}{Dublin (IE)}
%      {Participation with a talk on a Machine Learning application to Medical Physics during an international conference}
    %
%    \cvconfitem{Learning to Discover}
%      {2022}{Orsay (FR)}
%      {Participation with a talk on Machine Learning applications to Particle Physics during an international conference}
    %
%    \cvconfitem{107\textdegree\ Congresso Nazionale della SIF}
%      {2021}{online}
%      {\ul{Talk:} ``Simulating the LHCb detector with GANs''}
    %
%    \cvconfitem{First ML-INFN Hackathon}
%      {2021}{online}
%      {\ul{Theme:} Introduction to Machine Learning topics interesting for INFN (use-case selected: Medical Physics)}
    %
%    \cvconfitem{Workshop della Commissione Calcolo e Reti dell'INFN}
%      {2021}{online}
%      {\ul{Talk:} ``Simulating the LHCb detector with GANs''}
    %
%    \cvconfitem{1st CloudBank EU Workshop}
%      {2021}{online}
%      {\ul{Talk:} ``LHCb deployment in AWS''}
    %
%	\cvconfitem{Artificial Intelligence in Health and Well-Being}
%      {2019}{Firenze (IT)}
%      {\ul{Theme:} Artificial Intelligence applied to health and well-being}
    %
%\end{cvtable}

\vspace{1em}

% +--------------+
% |   OUTREACH   |
% +--------------+
\cvsection{Outreach \& Dissemination}
\vspace{-0.5em}

\begin{cvtable}[2]
    %
    \cvgenitem{Plenary speaker}
      {2024}{Fondazione Uniser Pistoia}
      {Invited plenary talk on Artificial Intelligence applications in Physics, delivered within a science outreach seminar series organized by Uniser Pistoia (Tuscany, Italy)}
    %
    [-0.3em]
    %
    \cvgenitem{Plenary speaker}
      {2023}{Festival della Scienza Alto Vicentino}
      {Invited plenary talk on Artificial Intelligence at a city-wide science outreach festival held in Schio (Veneto, Italy)}
    %
    [-0.3em]
    %
    \cvgenitem{Book author}
      {2021}{Sassi Junior}
      {Contributor, on behalf of INFN, to a children’s science book featuring a section dedicated to Artificial Intelligence}
    %
\end{cvtable}

%\vspace{1em}

% +------------+
% |   AWARDS   |
% +------------+
%\cvsection{Honors \& Awards}
%\vspace{-0.5em}

%\begin{cvtable}[2]
    %
%    \cvawitem{2025 Breakthrough Prize in Fundamental Physics}
%      {Breakthrough Prize Foundation}{2025}
%      {TBA}
    %
%    \cvawitem{``Giulia Vita Finzi'' award}
%      {INFN}{2022}
%      {National INFN award for the best Master thesis on computing and networks}
    %
%    \cvawitem{Ph.D. Scholarship in Smart Computing}
%      {INFN}{2020 -- 2023}
%      {Scholarship to explore and develop Artificial Intelligence techniques for Medical and Particle Physics}
    %
%	\cvawitem{Scholarship for research activity}
%      {INFN}{2019}
%      {National grant to pass three months at CERN for research activity}
	%
%	\cvawitem{Scholarship for thesis abroad}
%      {University of Firenze}{2017}
%	   {Local grant to pass ten days at CERN for Bachelor thesis}
	%
%\end{cvtable}

\cvsignature

\end{document} 
